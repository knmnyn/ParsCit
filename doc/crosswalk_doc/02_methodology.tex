Before getting into the details of the implementation, a brief description of the Omnipage output as well as PDFx output is provided in the following subsections.

\section{Omnipage Output XML}
The XML generated by Omnipage after processing a document has the following structure:
\begin{itemsize}
\item Each document is divided into pages.
  Pages are divided into sections that represent blocks of text (or tables, images etc) placed horizontally from top to bottom of the page.
  Each section is divided into columns.
  If the running text in eah sentence occupies the entire width of the page, then there would be just one column.
  However, it is very common for scientific articles to have two columns with the text first filling up the left column and then the right.
  Each column is divided into paragraphs, paragraphs into lines, lines into runs, runs into words.
\item Word represents the smallest modularity in the xml.
  The 'space' tag which appears after each word tag is also of the smallest modularity.
\item The order of text in the xml reflects the flow of text on a page.
  This means that the xml covers all the paragraphs, lines and words in the left column and then moves to the right column in case of 2 columns.
\item Runs appear within a line if there is a change in the textual property (font, sub/super-script etc) of words within that line.
  Each run represents a block of text that has the same textual properties.
\item Elements like tables, figures and the page number on a page appear within the 'dd' tag.
\end{itemsize}

\section{PDFx Output XML}
Following are the points regarding the structure of XML generated by PDFx after processing pdf documents.
\begin{itemsize}
\item The XML contains a group of 'fontspec' tags at the beginning of the document.
  Each 'fontspec' tag is for a font that has been observed by PDFx in that document and contains information about the font family, the size of text with that font and a unique identifier for the size and font family appearing in that document.
\item Each page in the pdf document has a corresponding 'page' tag. Unlike Omnipage, there are no further divisions to the page.
  Each page tag is essentially a dump of all the words that appear on that page.
\item Similar to the organization of text flow in Omnipage, the PDFx XMl also has all the words appearing in an order reflecting the flow of information in the page.

\section{Implementaiton Overview}

\section{Architecture}
